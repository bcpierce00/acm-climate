\newif\ifdraft\drafttrue   % comments / todos for now
\newif\iflater\laterfalse  % comments / todos for final version

\documentclass[12pt]{article}

\usepackage{fullpage}
\usepackage{graphicx}
\usepackage{hyperref}
\usepackage{multicol}

\newcommand{\bcp}[1]{\ifdraft{\bf [bcp: #1]}\fi}
\newcommand{\mwh}[1]{\ifdraft{\bf [mwh: #1]}\fi}
\newcommand{\COtwoE}{CO$_2$e}

\begin{document}

\title{\Huge ACM and Climate Change}
\author{Benjamin C. Pierce (University of Pennsylvania) \\
Michael Hicks (University of Maryland) \\
Crista Lopes (University California, Irvine) \\
Jens Palsberg (University of California, Los Angeles)}

\maketitle

% \begin{multicols}{3}

ACM's vibrant conference culture has a dark side: it contributes
significantly to climate change.

\begin{figure}[t]
\centering
\includegraphics[width=4in]{wmo_stripes.png}
\caption{Variation in annual global temperature, 1850 to 2018 (range from
  blue to red: $1.35^\circ$C) \cite{WarmingStripes}}
\label{fig:warming}
\end{figure}

\begin{figure}[t]
\centering
\includegraphics[width=2in]{IPCC-cover.png}
\caption{2018 IPCC report \cite{IPCC18}.  Key finding: Emissions must
  decline by at least 40 percent by 2030 and reach net zero by 2050, if we
  are to hold warming to 1.5 degrees.  There is ``no documented historic
  precedent'' for the transformation of the world economy needed to achieve
  this.  }
\label{fig:IPCC-cover}
\end{figure}

\begin{figure}[t]
\centering
\includegraphics[width=6in]{SIGPLAN-confs.png}
\caption{Carbon footprint per participant for travel to recent SIGPLAN
  conferences\iflater\bcp{Get better file from the other paper, or get
    better version of this one from Crista.}\fi} 
\label{fig:confs}
\end{figure}

A broad scientific consensus warns that the earth is warming due to human
emissions of carbon dioxide and other greenhouse gases (Figure
\ref{fig:warming}). This is not a problem we can leave to future
generations: we are up against it now. Indeed, the UN's Intergovernmental
Panel on Climate Change (IPCC) recently recommended a 40\% decrease in
emissions by 2030 to avoid irreversible damage
(Figure~\ref{fig:IPCC-cover}). Achieving reductions on this scale will
require urgent and sustained commitment at all levels of society, including
non-governmental communities such as universities, companies, and scientific
societies like ACM.

Why ACM? Because one of ACM's main activities is organizing international
conferences, which produce significant emissions of greenhouse gases,
particularly from air travel. For example, a single round-trip flight from
Philadelphia to Paris emits the equivalent of about 1.7 tons of carbon
dioxide (\COtwoE) per passenger~\cite{carboncalculator}\iflater\bcp{Trim
  citation?}\fi---a substantial fraction of the {\em total} yearly emissions
for an average resident of the US (16.5 tons) or Europe (7
tons)~\cite{emissions}.  \iflater\bcp{For comparison, ...  (put in some
  figures about the footprints of a few other things that people might think
  of reducing---driving, eating meat, upgrading their fridge...)}\fi{}
Moreover, these emissions have no near-term 
technological fix because jet fuel is difficult to replace with
renewable energy sources~\cite{elec-air}. 
\iflater\bcp{Trimmed this sentence: Reducing the \COtwoE{} footprint
of conferences will require a reduction in the footprint of the attendees'
air travel.}\fi

We are the members of an {\em ad hoc} Committee on Climate Change within
ACM's Special Interest Group on Programming Languages (SIGPLAN) that was
formed in 2016 to consider possible strategies for mitigating the emissions
of our conferences \cite{ClimateCommitteReport}. Among the ones we find 
attractive are making it easier to use carbon offsets to mitigate aviation
emissions for conferences, choosing conference locations with an eye to
minimizing emissions, co-locating or merging conferences, virtualizing
program committees, and supporting remote participation.

Buying carbon offsets to reduce the impact of air travel
\cite{CarbonOFfsetReport} is an ``easy win'' that conferences should
implement now.  The idea of carbon offsets is to pay somebody to plant
trees, capture methane (from landfills or cattle herds, for example), buy
cookstoves to replace inefficient open-fire cooking in poor communities, or
carry out other actions that remove greenhouse gases from the air or prevent
their emission. At the moment, the average air travel to an international
conference can be offset by paying around \$30 \cite{CarbonOFfsetReport}.
ACM could streamline the purchase of such offsets and make offsetting the
default by folding them into its conference registration fees.

However, carbon offsetting is only a short-term mitigation, since the
potential supply of offsets is far less than the potential
demand~\cite{SEI-Report}. We must also begin significantly reducing
emissions, which means reducing the distance traveled by the sum of all
conference participants.

One way to do this is to choose conference locations that are close to
the center of mass of their likely participants.  For example, the
SIGGRAPH conference on computer graphics
locates in Los Angeles every other
year---an easy call, as this is near both an international airport and a major hub
of the worldwide entertainment industry. For other research areas, the
effects of location choice are less obvious. For example,
Figure~\ref{fig:confs} plots the 
locations of the four major SIGPLAN conferences over the past five years,
colored according to the estimated average carbon footprint per
participant. The figure shows significant differences across
conferences and across locations---indeed, a factor of two difference
between the 
lowest and highest. Why is this? One reason might be that the locus of a
conference's population is closer to one location than another, and changing
locations reduces the total cost. Another might be the presence of a local
population that will only attend if the conference is very close to
them. Our preliminary analysis suggests it may be both (and that deeper
analysis is needed to understand the situation better). The larger question
is how to use this kind of information to choose the best locations, over
time, for a diffuse population.

Another natural idea is to co-locate or consolidate conferences.  This is
already standard practice in some communities---e.g., SIGGRAPH, SC, and CVPR
are all ``the'' conference for their areas.  More communities---for example,
the programming languages and security communities, each with four
main conferences every year---could consider following suit.  A different
twist on this idea would be to hold a single conference in two locations
simultaneously (e.g., Boston and Paris), with a video link between the
sites. Switching to virtual program committee meetings for conferences is
another obvious step.\iflater \mwh{data about surprising connection to conference's
  overall footprint?}\bcp{Or leave it for next round of revisions when we
  may have done more/better analysis?}\fi{}  Indeed, this switch is already
happening: we found in a recent poll of ACM SIG chairs that, across ACM, a
majority of the SIGs' main conferences now use online PC meetings rather
than in-person ones.

Another approach is allowing some conference attendees to participate
remotely, saving both emissions and travel time.  Indeed, SIGPLAN already
livestreams all its four main conferences, while SIGACCESS and SIGCHI have
experimented with telepresence robots at their main
conferences~\cite{CHI-remote,Neustaedter}. \iflater\bcp{Next sentence breaks
the flow a bit.  Cut it?}\fi{}Journal-first initiatives, which
make conference attendance optional, can also help.  Looking further
ahead, some communities are already experimenting with fully virtual
% \mwh{reality?}
% BCP: Seems redundant with "fully"
conferences~\cite{OS}.  ACM members can contribute to the
development of technologies for connecting people effectively at a distance.

Each of these strategies will have some beneficial effect on emissions.  
Each may also have downsides. For example, locating conferences
near their existing centers of mass may discourage holding them in
countries like India and China where the community is still developing,
while virtual PC meetings and remote conference attendance may reduce
opportunities for informal connections and mentoring.  Getting beyond easy,
short-term solutions and fat-trimming exercises requires balancing competing
concerns; this will demand significant  
discussion, compromise, careful analysis of data, and creative
experimentation.  
%
\iflater\bcp{Added remainder of paragraph.  I'm not sure the two sentences work
  together very well, but I can't decide which to keep!  (Mike likes them
  both.)}\fi%
It will also involve questioning some fundamental assumptions about how
research gets done---e.g., that more and bigger conferences are better, and
that personal success is (positively) correlated with time spent in
airports.
%
To thrive over the long haul, we need to begin imagining what a zero-carbon
professional society will look like.

The problem of climate change is too large and too urgent to leave to world
leaders: organizations at every level must play their part by addressing
their own contributions, raising awareness among their members, and
establishing new ways of 
doing business in the lower-carbon future that is coming. For ACM, this
especially means developing strategies for reducing the climate impact of
conference-related travel.
%
You can help accelerate this process in a number of ways: 
\begin{itemize}
\item {\bf Get involved} in raising awareness about this issue in the
organizations you are a part of.  For example, ask the organizers of the ACM
conferences you attend what they are doing about reducing or offsetting
emissions from participants' travel.
\item {\bf Analyze} your organization's carbon-emitting activities to inform a
discussion of how best to reduce its footprint.
\item {\bf Connect} with others that are engaging with these issues---in
your workplace, in your other communities, and within ACM (e.g., on the
{\tt acm-climate} mailing list~\cite{acm-climate}).
\end{itemize}

\bibliographystyle{unsrt}
\bibliography{local}

\iflater
\newpage
\section*{Permissions for figures}

[1] Creative Commons license

[2] ``Reproduction of a limited number of figures or short excerpts of IPCC material is authorized free of charge and without formal written permission provided that the original source is properly acknowledged, with mention of the name of the report, the publisher and the numbering of the page(s) or the figure(s).''

[3] Authors


\section*{More Discussions:}
\begin{itemize}
\item New comments from Judith Bishop:

1. Figure 3 is unclear. I did not understand it. Maybe it is missing a key, or the circles are too small. A rework would be very beneficial.

2. All articles benefit from headings. It reveals the structure of the paper at a glance and assists in avoiding repetition. Think about some headings. Have a look at some recent Viewpoints papers.

3. The title is rather opaque. The article is mostly about conferences, so maybe that should be in the title. Viewpoints likes catchy titles, so something more personal such as "Your conference trip and climate change" might get more readers.

4. The article does not mention the journal publication alternative to conferences. We all know that computer scientists fight fiercely for the status of their conference papers on their CVs. Maybe climate change will be the tipping point that will cause them to become more like their colleagues in other disciplines and publish in journals, with just one conference a year, to meet colleagues and catch up. From my side, I do think this aspect should be mentioned.

\item Old comments from Andrew Chien:
\begin{quote}
So, should we really be so conference intensive? (we seem to be far more so than other disciplines)

If we are going to have SIG conferences, as a staple of CS, do they actually need to be face to face?  (could
make the f2f ones less frequent)  What would the scientific impact be?

We seem to have wired into the economics of our conferences that "bigger attendance is better".  this supports
the fixed costs, makes the hotels happy, etc.  Can we make conferences where smaller in-person attendance, but
larger total engagement (whatever that means... perhaps virtual or after the fact interaction) is maximized?

What does a zero-carbon professional society look like?  (if you peek at my zccloud.cs.uchicago.edu page, you'll
see that I don't mean offset, but rather actual zero-incremental carbon emissions)

If you've already thought about all of these, then that's great.  If not, please do take them as a set of potential
challenges to consider (all voluntary... as I intend them only as suggestions).
\end{quote}
\item 
About Crista's figure:
BCP: Would be great to supplement with a more refined analysis of registrations for each of these conferences; then we could add a "center of mass" dot for each of the conferences to the map. This would still not tell the whole story, since it's possible (likely) that the registration totals would have been different if the conferences had been located differently. But it would help tell the story.
\item 
Also about Crista's figure:
\begin{itemize}
\item 
BCP: We should quantify what the red and blue dots mean -- i.e., give readers an idea what overall difference a change of location would make to carbon footprint.
\item Mike: The dot colors mean per-capita footprint. Could we make their size represent the total footprint? Far-flung places might be high per-capita, but lower overall.
\item Crista: The order by total footprint is very similar to the normalized one. Here are the 5 top offenders by that criteria: ICFP-Japan, POPL-LA, POPL-SD, SPLASH-Vancouver, PLDI-Barcelona. There are some changes in the order of the conferences in the middle. I hesitated which plot to draw (each has positives and negatives) but decided this one would be less problematic. I don't think we have space for two, and overlaying the 2 data series will be confusing...
\item BTW, there's a LOT more that can be said about this data... But I think that will distract from the main message of this article. We need to chose whether this 2-page article is about the data analysis of our conferences or about a call for action. Maybe we can write a data analysis paper, upload it on arXiv, and link to it from this article?
\item
For example: POPL-Paris was the absolute winner by all criteria: it had the highest number of participants, the second-lowest normalized carbon footprint, and was mid-point in total carbon footprint, even with all those participants.
This is just one tidbit; there are many more interesting facts in this data.
\end{itemize}
\item 
From Michael Coblenz:
What about research on energy-efficient computing? The impact of this research might be far more significant than that of mitigating our direct carbon emissions, and this is more in line with the core skills of the community.
Also, what about research on telepresence and virtual meetings? There's obviously a lot of work in this area, but we may need more in order to (a) let us effectively trade off options, e.g. virtual PC meeting vs in-person; (b) improve the state of the art to make remote meetings more effective.
Consider also discussing the diversity implications on travel. Reducing travel expectations might make the field more attractive to women, who may face higher social expectations to stay home (and are more likely to be single parents). Surely there is a good article to cite here. I found https://journals.sagepub.com/doi/abs/10.1177/0950017006066999 but maybe there is something better. I'm looking…
\item 
Consider citing the new IEA report as another piece of evidence that action is needed
\item 
From Sophia: Does the article need an abstract to "set the reader's expectations"? I expected that there would be a list of actions in the end, or recommended resolutions. Instead, the article is more about what some communities do, and where individuals may be involved.  Perhaps say: "In this article we give a short overview of how different SCM communities tackle this problem, what thought are relevant to it, and conclude with suggestions on how to get involved". It breaks the flow, but it clarifies what the article is about.
\item 
Is the citation for emissions by region CO2 or CO2e?
\item 
Is there a ``more authoritative'' source for price of carbon offsets than
our carbon report?
\end{itemize}
\item \item 
Check where we've used ``carbon'' (or CO2) and where ``greenhouse gases''.  Gloss CO2 or CO2e as appropriate.
\fi

% \end{multicols}
\end{document}
