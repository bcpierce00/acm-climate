\newif\ifdraft\drafttrue   % comments / todos for now
\newif\iflater\laterfalse  % comments / todos for final version

\documentclass[12pt]{article}

\usepackage{fullpage}
\usepackage{graphicx}
\usepackage{hyperref}
\usepackage{multicol}

\newcommand{\bcp}[1]{\ifdraft{\bf [bcp: #1]}\fi}
\newcommand{\mwh}[1]{\ifdraft{\bf [mwh: #1]}\fi}
\newcommand{\COtwoE}{CO$_2$e}

\newcommand{\SECTION}{\paragraph*}

\begin{document}

\title{\Huge Conferences in the Era of Expensive Carbon}
% \title{\Huge ACM Conferences and Cost of Carbon}
\author{Benjamin C. Pierce (University of Pennsylvania) \\
Michael Hicks (University of Maryland) \\
Crista Lopes (University California, Irvine) \\
Jens Palsberg (University of California, Los Angeles)}

\maketitle

% \begin{multicols}{3}

% \begin{figure}[t]
% \centering
% \includegraphics[width=4in]{wmo_stripes.png}
% \caption{Variation in annual global temperature, 1850 to 2018 (range from
%   blue to red: $1.35^\circ$C) \cite{WarmingStripes}}
% \label{fig:warming}
% \end{figure}

% \begin{figure}[t]
% \centering
% \includegraphics[width=2in]{IPCC-cover.png}
% \caption{2018 IPCC report \cite{IPCC18}.  Key finding: Emissions must
%   decline by at least 40 percent by 2030 and reach net zero by 2050, if we
%   are to hold warming to 1.5 degrees.  There is ``no documented historic
%   precedent'' for the transformation of the world economy needed to achieve
%   this.  }
% \label{fig:IPCC-cover}
% \end{figure}

A broad scientific consensus warns that human
emissions of carbon dioxide and other greenhouse gases are warming the earth.
% (Figure \ref{fig:warming}).
This is not a problem that can be left to future
generations: The UN's Intergovernmental
Panel on Climate Change (IPCC) says a 40\% decrease in
emissions is needed by 2030 to avoid irreversible damage~\cite{IPCC-report}.
% (Figure~\ref{fig:IPCC-cover}).
Achieving reductions on this scale will
require urgent and sustained commitment at all levels of society, including
not only city, state, and national governments, but also 
universities, companies, and scientific societies
like ACM. 


% A broad scientific consensus warns us that human
% emissions of carbon dioxide and other greenhouse gases are warming the earth.
% % (Figure \ref{fig:warming}).
% This is not a problem to be left to future
% generations: we are up against it now. The UN's Intergovernmental
% Panel on Climate Change (IPCC) says a 40\% decrease in
% emissions by 2030 is needed to avoid irreversible damage~\cite{IPCC-report}.
% % (Figure~\ref{fig:IPCC-cover}).
% Achieving reductions on this scale will
% require urgent and sustained commitment at all levels of society, including
% large organizations like universities, companies, and scientific societies
% like ACM. 

Indeed, scientific societies arguably have an especially important role to
play, since, for many of their members, travel to conferences may be a
substantial or even dominant part of their individual contribution to
climate change.  A single round-trip flight from Philadelphia to Paris emits
the equivalent of about 1.7 tons of carbon dioxide (\COtwoE) per
passenger~\cite{carboncalculator}\iflater\bcp{Trim citation?}\fi---a
significant fraction of the {\em total} yearly emissions for an average
resident of the US (16.5 tons) or Europe (7 tons)~\cite{emissions}.
\iflater\bcp{For comparison, ...  (put in some figures about the footprints
  of a few other things that people might think of reducing---driving,
  eating meat, upgrading their fridge...)}\fi{} Moreover, these emissions
have no near-term technological fix, since jet fuel is difficult to replace
with renewable energy sources~\cite{elec-air}.
% \iflater\bcp{Trimmed this sentence: Reducing the \COtwoE{} footprint
% of conferences will require a reduction in the footprint of the attendees'
% air travel.}\fi

How should ACM respond to these facts?  Two years ago, SIGPLAN (ACM's
Special Interest Group on Programming Languages) convened an ad hoc Climate
Committee\cite{ClimateCommittee} to research this question.  After
investigating many options~\cite{ClimateCommitteReport} and discussing
intensively with SIGPLAN members and conference organizers, ACM staff, and
other SIG leaders, we have two concrete proposals. First, all ACM
conferences should publicly {\em account} for CO2e emitted as a result of
putting them on. Second, ACM should put a \emph{price} on carbon in
conference budgets, to create pressure on organizers to reduce their
footprints.  

\SECTION{Mandate public accountability.}  

Our first proposal is simple: {\bf Every ACM-sponsored conference should
publicly report its carbon footprint.} Most conferences'
footprints will be dominated by participants air travel, but the accounting
should go beyond this to include air travel to in-person PC meetings and
estimated hotel and food CO2e footprints.

Public accounting will nudge conference organizers and attendees to change
their behavior. By analogy, chain restaurants in the US are required by law
to post calorie counts of food items on menu boards, and studies show that
enlightened customers order, on average, up to 50 fewer calories a day as a
result~\cite{menu}.

As a case in point, SIGPLAN has been considering how to reduce CO2e usage
over the past couple of years, informed by an accounting of its own CO2e
expenditures. This discussion has led the organizers of two of its flagship
conferences (POPL and ICFP) to switch from in-person to on-line program
committee meetings, joining a trend among other SIGs, most of which now use
virtual PC meetings for some or all of their flagship conferences. It has
also prompted several conferences to increase their investments in
livestreaming and video recording to support remote participation.

SIGPLAN has also explored how to mitigate a conference's CO2e emissions by
purchasing \emph{carbon offsets}~\cite{CarbonOFfsetReport}.
%
A carbon offset is sold by a vendor who uses the raised funds to finance an
activity that permanently avoids emitting some amount of greenhouse gases.
The veracity and permanence of this CO2e emission avoidance is certified by
a watchdog organization. (Typically, planting trees is not a certifiable
activity, since it's hard to guarantee the trees won't be cut down, but
installing methane capture devices on landfill sites or buying
fuel-efficient stoves to replace open-fire cooking in poor communities can
be.)
%
Many organizations, including companies such as Google, Dell, Microsoft,
General Motors, Delta Airlines, Lyft, and Expedia, numerous universities,
and some academic societies, now buy carbon offsets in order to reduce their
net footprint. ACM conferences can do the same, and the purchases can be
included in the public accounting we are proposing.  In fact, ACM recently
decided that conferences could offer attendees the option to pay for an
offset during registration; the conference then uses the designated funds to
purchase the offset from a qualified vendor.  Some conferences are also
experimenting with corporate sponsorship of bulk offsets for the whole
conference.\footnote{We acknowledge that carbon offsets are a somewhat
  controversial 
  tool~\cite{anderson2012inconvenient,
  carbon-offsets-are-not-our-get-out-jail-free-card}.  Besides issues with
  verifiability and permanence, some observers worry that---especially given
  their currently artificially low cost---they will encourage complacency,
  leading organizations to postpone making the difficult changes that will
  be required to sustainably decarbonize society.  We believe that offsets
  should be considered as a useful short-term expedient, to be considered in
  parallel with, not instead of, more difficult cutting.}

Beyond encouraging conferences to take these sorts of simple steps, public
accounting of conference emissions will generate data that future organizers
can use.  To this end, emissions data should be published both in aggregate
form (total and per-participant footprint), for quick reference, and in raw
form (a suitably anonymized record for each individual attending each
conference, including originating city), and it should be made available in
a central place and in a uniform format, to facilitate analysis.

ACM should provide tools for gathering and publicizing this data. As a
start, SIGPLAN recently worked with ACM's SIG Governing Board to develop an
air-travel-focused CO2e calculator for conferences. Users can upload
conference registration data, and the calculator will estimate the CO2e cost
of air travel for those sufficiently far away. This calculator could be
expanded to account for other forms of conference emissions.


\SECTION{Put a Price on Carbon.}

While publicly accounting for CO2e expenditures will incentivize some CO2e
reductions, it is not likely, by itself, to induce major changes.
%
Science is a fundamentally social process, and the conference system
accelerates scientific research through high-bandwidth interaction,
direct dissemination of results, network building, and serendipitous
cross-fertilization.
%
Organizers and attendees will be reticent to make changes that might impact
these benefits.

\begin{figure}[t]
\centering
\includegraphics[width=6in]{SIGPLAN-confs2.png}
\includegraphics[width=6in]{ParticipantsOriginAll2.png}
% \includegraphics[width=6in]{Distances.png}
\caption{Top: Carbon footprint per participant for travel to recent SIGPLAN
  conferences\iflater\bcp{Get better file from the other paper, or get
    better version of this one from Crista.}\fi.  (The smallest dot (ICFP
  14, in Gothenburg, Sweden) represents .9 tons of CO2e per participant; the
  largest dot (ICFP 16, in Nara Japan) represents \bcp{1.94} tons per
  participant.) Bottom: Breakdown of continent-of-origin for participants in
  each conference.
% Bottom: Distances traveled by participants, one way, in Km; middle dashed
% line is the median travel distance. 
}
\label{fig:confs}
\end{figure}

The two visualizations of registration data from some recent conferences in
Figure~\ref{fig:confs} illustrate this difficulty.  The top diagram shows an
estimated per-participant CO2e footprint for each instance of the four major
SIGPLAN conferences over the past ten years (excluding a few for which we
had trouble getting data), with larger dots representing more emissions.
Eyeballing this diagram, it appears that carbon-conscious conference
planners should hold all of these conferences either in the northeast US or
in western Europe every year.  But the bottom visualization tells a
different story.  Here, the horizontal colored bar at the bottom represents
the continent on which each conference was held, and each vertical bar gives
a breakdown of the participants in that conference, colored according to the
continent they gave as their home address.  Again, from a superficial glance
at the colors, it is clear that, though a minority come from far away, the
majority of participants in each case are from the region where the
conference is being held---that is, there is a large amount of local
participation.\iflater\bcp{That point could be strengthened by adding some
  numbers (just in the text, not a figure) about what percentage of
  conference attendees attend even one additional instance of the same
  conference...}\fi{} From this, one might conclude that always locating
conferences in the same one or two places would significantly damage the
research community by discouraging participation from other parts of the
world---indeed, perhaps conferences should move around as much as
possible. 
%
%
% \bcp{Explain the two diagrams.  Discuss how they can be used to draw
%   contradictory conclusions.  E.g., one of the many footprint-reduction
%   measures that have been discussed in SIGPLAN~\cite{ClimateCommitteReport}
%   is locating conferences near their historical ``center of mass.''  Looking
%   at the top diagram, one might thus be tempted to locate all conferences in
%   the northeast US or western Europe.  But the bottom figure shows that much
%   of the participation in each instance of a conference is local; always
%   holding a conference at its ``center of mass'' will damage local
%   communities everywhere else.  (Maybe this would be a good place to refer
%   to~\cite{ClimateCommitteReport}---we can enumerate some of the
%   alternatives that one might have hoped the data would help us choose
%   between.)}  
% %
% \bcp{Maybe we can replace the bottom part of the figure by some clearer way
%   of visualizing the observation that much of the attendance at each
%   conference is local?  E.g., what if, instead of color-coding the different
%   sections of the bars by continent, we instead color-coded them by distance
%   traveled (in buckets of 500 miles, say)?}
%
These disparate conclusions suggest that reducing conference emissions will
require genuinely painful compromises.  The impulse to avoid thinking about
it is entirely understandable!

However, the present trajectory of world emissions is unsustainable, and
hard choices will have to be made if ACM is to play its part by reducing its
own emissions. How do we motivate organizers to make them?

This dilemma is a microcosm of the one that all of society faces. To address
it, many policy experts advocate some form of ``carbon pricing'' (either
taxes or cap-and-trade schemes), which imposes a concrete, immediate cost on
emissions~\cite{carbonprice}. Doing so makes manifest the hidden
environmental cost of CO2e emissions, thus incentivizing CO2e-reducing
changes without specifying exactly which, allowing for creative and
efficient responses. (Continuing the junk-food analogy, some municipalities
such as New York city and Philadelphia have imposed a per-calorie tax on
soft drinks, with some studies finding that doing so indeed reduced
consumption~\cite{sodatax}.)

Ideally, at some point, governments will impose carbon pricing uniformly,
and all CO2e-intensive activities will have to pay it. But ACM can send a
strong message about the importance of this issue by acting now. In
particular, as several companies have shown, even an artificial ``internal
price'' on carbon creates a significant price signal that shapes behavior.
%
Thus, our second proposal is that {\bf ACM should impose a surcharge on
  conferences based on their carbon footprint.}  The charge should be low at
the beginning and increase steadily (and predictably) year on year.
%
Conference organizers can then choose (on the basis of careful analysis of
data and weighing of benefits) how best to balance their budgets---whether
by decreasing per-participant emissions, decreasing (physical)
participation, increasing registration fees, soliciting additional corporate
sponsorship, or other means.  \iflater\mwh{Problem: We showed just above
  that it's hard to figure out what to do. Come back to this point here and
  say that the surcharge will spur bold choices?  Maybe say that these
  choices won't necessarily be bad, but could be really great?}\fi

What should ACM do with the funds collected from this surcharge?  One can
imagine many good uses, including carbon offsetting, supporting ``green''
computing research (for example, what if ACM funded a major cross-cutting
research initiative specifically aimed at understanding how to best replace
or approximate the socializing and networking aspects of conferences in a
virtual setting?\iflater\bcp{I'm tempted to make this into a third
  proposal!}\fi) \iflater\bcp{can we mention some more specifics?}\fi, and
defraying the costs of virtualizing conferences (livestreaming, etc.).
\iflater\bcp{Expand this paragraph a bit.}\fi

Beyond encouraging conferences to trim their emissions, putting a concrete,
visible price on 
carbon will stimulate creative rethinking of the conference model
itself---for example, seriously considering alternatives such as
rapid-turnaround journal-only publishing models, once-a-year
megaconferences, and entirely virtual conferences~\cite{NCN}.
%
(A potential sticking point here is that some of these strategies will also
significantly reduce conference revenues; in the limit, this may impact the
income stream of ACM itself, which could make emissions reduction
politically unpalatable unless ACM's conference-focused business model is
also adjusted.  Planning for this eventuality should begin now!)

% \SECTION{Prepare for Deeper Changes}

% Our two concrete proposals aim to make conferences much more
% carbon-efficient, allowing ACM to play its part in the 40\% reduction in
% emissions that is needed over the next decade.  But this is just a first
% step.  Looking further ahead, the challenge will be to eliminate {\em all}
% greenhouse emissions by around 2050 and then move into global negative
% emissions during the second half of the century.

% \bcp{I wish we could put a paragraph here about the long-term picture and
%   the need to rethink scientific exchange and ACM's business model.  That
%   will please our grumpy reviewer, and I think it's actually a point worth
%   making.\cite{NCN}} 


\SECTION{Conclusion}

% Ultimately, the carbon cost of air travel may get so high, and individual
% carbon budgets so low, that people simply stop flying except on rare
% occasions.  

The climate crisis is too important to ignore and too urgent to leave for
world leaders to address at their own pace: Organizations at every scale
must confront their own contributions, raise awareness among their members,
and begin establishing new ways of doing business in the lower-carbon future
that is coming.  ACM should lead the way!


%%%%%%%%%%%%%%%%%%%%%%%%%%%%%%%%%%%%%%%%%%%%%%%%%%%%%%%%%%%%%%%%%%%%%%%%%%%%%

\newpage

\section*{Old text (that may be useful to mine)}

\SECTION{Putting a Price on Carbon}

Exactly how the price of carbon
should be set is a challenging policy question, but it has a clear goal: It
should be chosen so that the world arrives at zero emissions within the
timeframe that climate scientists tell us is safe.

To be effective in the long term, a carbon price needs to be uniform across
society---i.e., it should be imposed by governments.  Indeed, some 40
countries and more than 20 cities and states have already adopted
carbon pricing policies~\cite{pricingcarbon}; more will follow.
%
Bottom line: We are entering an era in which carbon will be expensive.

Anticipating this era, what should we rationally do now to ensure that 
computer-science research continues to thrive in a world
where air travel is more expensive?  Believing that it needs to come quickly
to avert disaster, what can we do to accelerate the process? \mwh{The
  question now seems to have changed. The paper started by saying ACM has an
  ``especially important role to play,'' but now seems to suggest it should
  be reacting to an inevitable/coming future. These questions probably need
  to go much earlier, and the earlier text reworked around them.}
\bcp{I tried rewording the opening of the second paragraph to address this.}

\SECTION{Starting Now}

One possible short-term response to both questions is to use {\em
  carbon offsets} to voluntarily impose a price on our own
emissions~\cite{CarbonOFfsetReport}.
%
In a nutshell, the idea is that a conference raise funds from
registration fees or corporate sponsors and donates them, via a carbon offset
{\em vendor}, to a {\em provider} who  uses them to support an activity
(e.g., installing methane capture devices on landfill sites or buying
fuel-efficient stoves to replace open-fire cooking in poor communities) that
is {\em certified} (by yet another organization) to permanently avoid
emitting an amount of greenhouse gases equivalent to the emissions from
participants' travel to the conference.
%
Many organizations (including companies such as Google, Dell, Microsoft,
General Motors, Delta Airlines, Lyft, and Expedia, numerous universities,
and a number of academic societies) are already buying carbon offsets, and
some ACM conferences have already started to experiment with voluntary
carbon offset fees at registration time.

Carbon offsets also raise some significant concerns.
% %
% A superficial one is that most organizations do not currently have
% accounting procedures in place to deal with them, and the piecemeal
% political and bureaucratic work of designing and implementing a separate
% policy for each organization can consume significant energy.
%
One is that the long-term climate benefit of some offsetting strategies (for
example, paying people to plant trees) is controversial: it is important to
choose reputable vendors and projects (such as methane capture) whose
net benefit is clear.
%
A more fundamental problem is that it is hard to decide how
much one should pay to offset a ton of carbon: current prices vary
around an average of around \$15 per ton, depending on the vendor, the
nature of the offset project, the region of the world where it will be
implemented, etc.  Moreover, in order for offsets (or any other form of a
carbon pricing) to fully motivate the needed changes in behavior, the price
may need to go much higher---the true ``social cost'' of a ton of carbon is
variously estimated at anywhere from \$40 to \$400 per
ton~\bcp{citation}---and setting it too low may encourage complacency.
%
While acknowledging these concerns, we believe that voluntary
carbon offsetting can be an effective near-term tool for raising awareness
and beginning to shift organizational behavior because it associates
emissions with a number, giving decision makers within the organization
something concrete to optimize.

\SECTION{Grounding the Discussion}

\begin{figure}[t]
\centering
\includegraphics[width=6in]{SIGPLAN-confs.png}
\includegraphics[width=6in]{ParticipantsOriginAll.png}
\caption{Top: carbon footprint per participant for travel to recent SIGPLAN
  conferences\iflater\bcp{Get better file from the other paper, or get
    better version of this one from Crista.}\fi Bottom: origin of participants for each conference.} 
\label{fig:confs}
\end{figure}

What, concretely, would it mean to ``optimize'' the carbon footprint of
conferences?
%
% Beyond imposing a price on our carbon footprint (or waiting for one
% to be imposed from outside), a sane response to climate change should
% clearly also include taking steps to reduce the size of this
% footprint.
% \mwh{Don't like previous sentence. It seems to have gotten away from
%   the point of the paper, which is using price to change balance of
%   apples to oranges. Analyzing data is useful for figuring out ACM's
%   internal price, not tweaking around the edges, which is how it reads
%   below.} 
One might contemplate a wide array of options~\cite{ClimateCommitteReport},
from easy-but-incremental changes like making program committee meetings
virtual instead of physical and enabling virtual participation via
live-streaming, all the way to complete re-imagining of conferences and
their role in science (holding conferences completely in virtual spaces,
replacing conferences with journals, etc.), with some intriguing hybrid
forms in the middle (regional conferences, mega-conferences, conferences
held simultaneously at two locations on different continents, etc.).
%
How much of a difference would these make to emissions?  

% But before undertaking significant changes, we need to understand the
% situation better.  For most of these ideas, it is hard to confidently
% predict either the potential benefits (in terms of reduced emissions) or the
% potential costs (to scientific exchange, career development of young
% scientists, etc.).  
A preliminary analysis of registration data from
a number of SIGPLAN conferences paints a rather complicated picture.
%
\bcp{Put the discussion of Figure 1 (and an accompanying figure
showing distance traveled to several instances of the same conference) here;
this illustrates the difficulty and the need for careful data gathering and
analysis.} 

All parts of ACM should be doing this kind of analysis.  (The acm-climate
mailing list is one place to exchange ideas about this.)

\SECTION{Plan Ahead}

Looking further ahead, the expectation of expensive carbon invites a deeper
rethinking of how academic computer science is done.  
%
\bcp{
\begin{itemize}
\item Add a bit of meat about how the current economic model may be
  unsustainable and reworking it will involve substantial organizational
  changes.  (There may be some good words that we can paraphrase from our
  second reviewer!)
\item monitoring and mitigating carbon costs should be an ACM-level activity
\end{itemize}
}

The climate crisis is too large and too urgent to leave to world leaders to
address at their own pace: organizations at every scale should confront
their own contributions, raise awareness among their members, and begin
establishing new ways of doing business in the lower-carbon future that is
coming.
% For ACM, this especially means confronting the carbon-intensive
% nature of our conferences.
% %
% You can help accelerate this process in a number of ways: 
% \begin{itemize}
% \item {\bf Get involved} in raising awareness about this issue in the
% organizations you are a part of.  \bcp{Does the rest of this smack of
%   incrementalism?  Maybe delete it.  ``For example, ask the organizers of
%   the conferences you attend what they are doing about reducing or
%   offsetting emissions from participants' travel.''}
% \item {\bf Connect} with others that are engaging with these issues---in
% your workplace, in your other communities, and within ACM (e.g., on the
% {\tt acm-climate} mailing list~\cite{acm-climate}).
% \item {\bf Analyze} your organization's carbon-emitting activities to
% understand how to reduce its footprint now and prepare for the coming era of
% expensive carbon.
% \end{itemize}

\bcp{New outline:
\begin{itemize}
\item Accountability.
  \begin{itemize}
  \item Proposal: Every ACM-sponsored conference should be required to
  publicly report its carbon footprint (total and per-participant).
  (Including PC meetings, participant travel, maybe hotel?).
  \item ACM needs to provide tools to help conferences do this.  (SIGPLAN
  built one.)
  \item Benefits:
  \begin{itemize}
  \item Encourages people to pay attention, which will lead to some changes
  in behavior all by itself. 
  \item Generates data that will be needed for grounding discussions of how
  optimize footprint.  E.g., SIGPLAN has mostly gone to virtual PC
  meetings.  Livestreaming.
  \item Indeed, optimizing footprint is not so easy.  (Describe some of our
  data.)  More data and thorough analysis is needed.
  \item Segue: Colocation, etc.
  \end{itemize}
  \end{itemize}
\item Pricing carbon.
  \begin{itemize}
  \item To motivate more painful changes and force people to make decisions,
  we need a price on carbon.
  \item Ultimately, we want/expect governments to do this.  But we don't need to
  wait. 
  \item Proposal: ACM should apply downward pressure on conferences by
  charging conferences a surcharge based on their carbon footprint.
  \item This money can be used in many ways: buying carbon offsets,
  supporting ``green research'', supporting virtualizing conferences
  (livestreaming, etc.), supporting a transition to a low-carbon / few
  conferences business model for ACM.
  \end{itemize}
\end{itemize}
}

\bcp{Things still missing:
  \begin{itemize}
  \item citations for the claims we make (and the carbon-offsets-are-bad paper
  mentioned by the reviewer)
  \item Go through all the comments and ideas below (including the CACM
  reviews) one last time
  \item Find a place for this: {\tt acm-climate} mailing list~\cite{acm-climate}
  \item Think about adding headings, per Bishop's recommendation

  \item Write a cover letter
  \begin{itemize}
  \item We tried hard to remove all incrementalism and make it much more
  hard hitting
  \item However, we did keep the suggestion of carbon offsets, hopefully
  in a more convincing / palatable / better argued form
  \item what else?
  \end{itemize}
  \end{itemize}
}

\bibliographystyle{unsrt}
\bibliography{local}

\iflater
\newpage
\section*{Permissions for figures}

[1] Creative Commons license

[2] ``Reproduction of a limited number of figures or short excerpts of IPCC material is authorized free of charge and without formal written permission provided that the original source is properly acknowledged, with mention of the name of the report, the publisher and the numbering of the page(s) or the figure(s).''

[3] Authors

\section*{Old Text}

We are the members of an {\em ad hoc} Committee on Climate Change within
ACM's Special Interest Group on Programming Languages (SIGPLAN) that was
formed in 2016 to consider possible strategies for mitigating the emissions
of our conferences \cite{ClimateCommitteReport}. Among the ones we find 
attractive are making it easier to use carbon offsets to mitigate aviation
emissions for conferences, choosing conference locations with an eye to
minimizing emissions, co-locating or merging conferences, virtualizing
program committees, and supporting remote participation.

Buying carbon offsets to reduce the impact of air travel
\cite{CarbonOFfsetReport} is an ``easy win'' that conferences should
implement now.  The idea of carbon offsets is to pay somebody to plant
trees, capture methane (from landfills or cattle herds, for example), buy
cookstoves to replace inefficient open-fire cooking in poor communities, or
carry out other actions that remove greenhouse gases from the air or prevent
their emission. At the moment, the average air travel to an international
conference can be offset by paying around \$30 \cite{CarbonOFfsetReport}.
ACM could streamline the purchase of such offsets and make offsetting the
default by folding them into its conference registration fees.

However, carbon offsetting is only a short-term mitigation, since the
potential supply of offsets is far less than the potential
demand~\cite{SEI-Report}. We must also begin significantly reducing
emissions, which means reducing the distance traveled by the sum of all
conference participants.

One way to do this is to choose conference locations that are close to
the center of mass of their likely participants.  For example, the
SIGGRAPH conference on computer graphics
locates in Los Angeles every other
year---an easy call, as this is near both an international airport and a major hub
of the worldwide entertainment industry. For other research areas, the
effects of location choice are less obvious. For example,
Figure~\ref{fig:confs} plots the 
locations of the four major SIGPLAN conferences over the past five years,
colored according to the estimated average carbon footprint per
participant. The figure shows significant differences across
conferences and across locations---indeed, a factor of two difference
between the 
lowest and highest. Why is this? One reason might be that the locus of a
conference's population is closer to one location than another, and changing
locations reduces the total cost. Another might be the presence of a local
population that will only attend if the conference is very close to
them. Our preliminary analysis suggests it may be both (and that deeper
analysis is needed to understand the situation better). The larger question
is how to use this kind of information to choose the best locations, over
time, for a diffuse population.

Another natural idea is to co-locate or consolidate conferences.  This is
already standard practice in some communities---e.g., SIGGRAPH, SC, and CVPR
are all ``the'' conference for their areas.  More communities---for example,
the programming languages and security communities, each with four
main conferences every year---could consider following suit.  A different
twist on this idea would be to hold a single conference in two locations
simultaneously (e.g., Boston and Paris), with a video link between the
sites. Switching to virtual program committee meetings for conferences is
another obvious step.\iflater \mwh{data about surprising connection to conference's
  overall footprint?}\bcp{Or leave it for next round of revisions when we
  may have done more/better analysis?}\fi{}  Indeed, this switch is already
happening: we found in a recent poll of ACM SIG chairs that, across ACM, a
majority of the SIGs' main conferences now use online PC meetings rather
than in-person ones.

Another approach is allowing some conference attendees to participate
remotely, saving both emissions and travel time.  Indeed, SIGPLAN already
livestreams all its four main conferences, while SIGACCESS and SIGCHI have
experimented with telepresence robots at their main
conferences~\cite{CHI-remote,Neustaedter}. \iflater\bcp{Next sentence breaks
the flow a bit.  Cut it?}\fi{}Journal-first initiatives, which
make conference attendance optional, can also help.  Looking further
ahead, some communities are already experimenting with fully virtual
% \mwh{reality?}
% BCP: Seems redundant with "fully"
conferences~\cite{OS}.  ACM members can contribute to the
development of technologies for connecting people effectively at a distance.

Each of these strategies will have some beneficial effect on emissions.  
Each may also have downsides. For example, locating conferences
near their existing centers of mass may discourage holding them in
countries like India and China where the community is still developing,
while virtual PC meetings and remote conference attendance may reduce
opportunities for informal connections and mentoring.  Getting beyond easy,
short-term solutions and fat-trimming exercises requires balancing competing
concerns; this will demand significant  
discussion, compromise, careful analysis of data, and creative
experimentation.  
%
\iflater\bcp{Added remainder of paragraph.  I'm not sure the two sentences work
  together very well, but I can't decide which to keep!  (Mike likes them
  both.)}\fi%
It will also involve questioning some fundamental assumptions about how
research gets done---e.g., that more and bigger conferences are better, and
that personal success is (positively) correlated with time spent in
airports.
%
To thrive over the long haul, we need to begin imagining what a zero-carbon
professional society will look like.


\section*{More Discussions:}
\begin{itemize}
\item Our notes:

Remove the IPCC image and mark the other “uninformative” image as up to the design department (i.e., we’re happy to keep it or drop it or change it to something else)

The map of registration data should be improved at least by reformatting and possibly by better data analysis, if we can get any of that done in time.  I need to schedule a meeting with Aalok and Yannick about that.

Some of the more “incremental” ideas can be removed — we included some that we ourselves were not very excited about.

We can make clearer that carbon offsets are controversial, and why

We can use the space gained by some of the above to add some discussion at the end about:

Why it’s important that ACM be serious about making serious cuts (i.e., because it gives us moral authority and sets a good example for other parts of society), even though in some sense anything that we (or any individual or “small” group does is a drop in the ocean).

The fact that ACM should also be thinking further ahead, which may mean fundamentally questioning its entire business model, along with a call to start this conversation.

\item New comments from Judith Bishop:

1. Figure 3 is unclear. I did not understand it. Maybe it is missing a key, or the circles are too small. A rework would be very beneficial.

2. All articles benefit from headings. It reveals the structure of the paper at a glance and assists in avoiding repetition. Think about some headings. Have a look at some recent Viewpoints papers.

3. The title is rather opaque. The article is mostly about conferences, so maybe that should be in the title. Viewpoints likes catchy titles, so something more personal such as "Your conference trip and climate change" might get more readers.

4. The article does not mention the journal publication alternative to conferences. We all know that computer scientists fight fiercely for the status of their conference papers on their CVs. Maybe climate change will be the tipping point that will cause them to become more like their colleagues in other disciplines and publish in journals, with just one conference a year, to meet colleagues and catch up. From my side, I do think this aspect should be mentioned.

\item Old comments from Andrew Chien:
\begin{quote}
So, should we really be so conference intensive? (we seem to be far more so than other disciplines)

If we are going to have SIG conferences, as a staple of CS, do they actually need to be face to face?  (could
make the f2f ones less frequent)  What would the scientific impact be?

We seem to have wired into the economics of our conferences that "bigger attendance is better".  this supports
the fixed costs, makes the hotels happy, etc.  Can we make conferences where smaller in-person attendance, but
larger total engagement (whatever that means... perhaps virtual or after the fact interaction) is maximized?

What does a zero-carbon professional society look like?  (if you peek at my zccloud.cs.uchicago.edu page, you'll
see that I don't mean offset, but rather actual zero-incremental carbon emissions)

If you've already thought about all of these, then that's great.  If not, please do take them as a set of potential
challenges to consider (all voluntary... as I intend them only as suggestions).
\end{quote}
\item 
About Crista's figure:
BCP: Would be great to supplement with a more refined analysis of registrations for each of these conferences; then we could add a "center of mass" dot for each of the conferences to the map. This would still not tell the whole story, since it's possible (likely) that the registration totals would have been different if the conferences had been located differently. But it would help tell the story.
\item 
Also about Crista's figure:
\begin{itemize}
\item 
BCP: We should quantify what the red and blue dots mean -- i.e., give readers an idea what overall difference a change of location would make to carbon footprint.
\item Mike: The dot colors mean per-capita footprint. Could we make their size represent the total footprint? Far-flung places might be high per-capita, but lower overall.
\item Crista: The order by total footprint is very similar to the normalized one. Here are the 5 top offenders by that criteria: ICFP-Japan, POPL-LA, POPL-SD, SPLASH-Vancouver, PLDI-Barcelona. There are some changes in the order of the conferences in the middle. I hesitated which plot to draw (each has positives and negatives) but decided this one would be less problematic. I don't think we have space for two, and overlaying the 2 data series will be confusing...
\item BTW, there's a LOT more that can be said about this data... But I think that will distract from the main message of this article. We need to chose whether this 2-page article is about the data analysis of our conferences or about a call for action. Maybe we can write a data analysis paper, upload it on arXiv, and link to it from this article?
\item
For example: POPL-Paris was the absolute winner by all criteria: it had the highest number of participants, the second-lowest normalized carbon footprint, and was mid-point in total carbon footprint, even with all those participants.
This is just one tidbit; there are many more interesting facts in this data.
\end{itemize}
\item 
From Michael Coblenz:
What about research on energy-efficient computing? The impact of this research might be far more significant than that of mitigating our direct carbon emissions, and this is more in line with the core skills of the community.
Also, what about research on telepresence and virtual meetings? There's obviously a lot of work in this area, but we may need more in order to (a) let us effectively trade off options, e.g. virtual PC meeting vs in-person; (b) improve the state of the art to make remote meetings more effective.
Consider also discussing the diversity implications on travel. Reducing travel expectations might make the field more attractive to women, who may face higher social expectations to stay home (and are more likely to be single parents). Surely there is a good article to cite here. I found https://journals.sagepub.com/doi/abs/10.1177/0950017006066999 but maybe there is something better. I'm looking…
\item 
Consider citing the new IEA report as another piece of evidence that action is needed
\item 
From Sophia: Does the article need an abstract to "set the reader's expectations"? I expected that there would be a list of actions in the end, or recommended resolutions. Instead, the article is more about what some communities do, and where individuals may be involved.  Perhaps say: "In this article we give a short overview of how different SCM communities tackle this problem, what thought are relevant to it, and conclude with suggestions on how to get involved". It breaks the flow, but it clarifies what the article is about.
\item 
Is the citation for emissions by region CO2 or CO2e?
\item 
Is there a ``more authoritative'' source for price of carbon offsets than
our carbon report?
\end{itemize}
\item 
Check where we've used ``carbon'' (or CO2) and where ``greenhouse gases''.  Gloss CO2 or CO2e as appropriate.
\fi

% \end{multicols}
\end{document}
