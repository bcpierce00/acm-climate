\newif\ifdraft\drafttrue   % comments / todos for now
\newif\iflater\laterfalse  % comments / todos for final version

\documentclass[12pt]{article}

\usepackage{fullpage}
\usepackage{graphicx}
\usepackage{hyperref}
\usepackage{multicol}

\newcommand{\bcp}[1]{\ifdraft{\bf [bcp: #1]}\fi}
\newcommand{\mwh}[1]{\ifdraft{\bf [mwh: #1]}\fi}
\newcommand{\COtwoE}{CO$_2$e}

\newcommand{\SECTION}{\paragraph*}

\begin{document}

\title{\Huge Conferences in an Era of Expensive Carbon}
% \title{\Huge ACM Conferences and Cost of Carbon}
\author{Benjamin C. Pierce (University of Pennsylvania) \\
Michael Hicks (University of Maryland) \\
Crista Lopes (University California, Irvine) \\
Jens Palsberg (University of California, Los Angeles)}

\maketitle

% ACM's vibrant conference culture has a dark side: it contributes
% significantly to climate change.

A broad scientific consensus warns us that human
emissions of carbon dioxide and other greenhouse gases are warming the earth.
% (Figure \ref{fig:warming}).
This is not a problem we can leave to future
generations: The UN's Intergovernmental
Panel on Climate Change (IPCC) says a 40\% decrease in
emissions by 2030 is needed to avoid irreversible damage~\cite{IPCC-report}.
% (Figure~\ref{fig:IPCC-cover}).
Achieving reductions on this scale will
require urgent and sustained commitment at all levels of society, including
not only city, state, and national governments, but also 
universities, companies, and scientific societies
like ACM. 

Indeed, scientific societies have an especially important role to
play, since one of their main activities is organizing international
conferences, which produce significant greenhouse gas emissions,
particularly from air travel. A single round-trip flight from
Philadelphia to Paris emits the equivalent of about 1.7 tons of carbon
dioxide (\COtwoE) per passenger~\cite{carboncalculator}\iflater\bcp{Trim
  citation?}\fi---a substantial fraction of the {\em total} yearly emissions
for an average resident of the US (16.5 tons) or Europe (7
tons)~\cite{emissions}.  \iflater\bcp{For comparison, ...  (put in some
  figures about the footprints of a few other things that people might think
  of reducing---driving, eating meat, upgrading their fridge...)}\fi{}
Moreover, these emissions have no near-term 
technological fix, jet fuel being difficult to replace with
renewable energy sources~\cite{elec-air}. 
% \iflater\bcp{Trimmed this sentence: Reducing the \COtwoE{} footprint
% of conferences will require a reduction in the footprint of the attendees'
% air travel.}\fi

Motivated by this situation, in 2016 we formed the ACM SIGPLAN Climate
Committee. Since then, we made presentations at SIGPLAN events, spoken with
ACM and SIG leaders, and assembled a report assessing a number of ideas for
reducing the \COtwoE{} costs of conference
activities~\cite{ClimateCommitteReport}.  Our report considers a range of
options, fromn easy-but-incremental changes like making program committee
meetings virtual instead of physical and enabling virtual participation in
conferences via live-streaming, all the way to complete
re-conceptualizations of conferences and their role in science (e.g.,
replacing conferences with journals, holding conferences completely in
virtual spaces, etc.), as well as some intriguing points in the middle of
the spectrum (regional conferences, mega-conferences, conferences held
simultaneously at two locations on different continents, etc.).

\SECTION{The Challenge of CO2 Reduction}

Those we spoke to have been almost universally supportive of reducing
\COtwoE{} emissions due to conferences, but very few are eager to
fundamentally change the way conferences are organized. This hesitation
makes sense. Science is a fundamentally social process, and the conference
system accelerates research through high-bandwidth interaction, direct
dissemination of results, network building, and serendipitous
cross-fertilization.  So how can we weight the ennvironmental benefit of a
particular \COtwoE-reducing change, such as on-line PC meetings, against the
harm the change could cause to the scientific process, such as the lost
opportunity to build networks and germinate synergistic ideas?

Scientific value and environmental harm are apples and oranges---hard to
compare.  Indeed, they are even hard to measure. In particular, while we can
reasonably estimate the \COtwoE{} directly saved by a particular change, it
is much harder to assess the impact of the example it sets.

This impact is important---perhaps more important than \COtwoE{} reductions
themselves. Emissions from air travel to conferences are the lion's share of
ACM's \COtwoE{} footprint, but air travel only contributes 2\% of current
global emissions~\cite{emissions}. But this small fraction should not be
taken as an excuse for inaction: the fact that any individual's or
organization's share of global emissions is miniscule does not mean it
doesn't matter. {\em No} individual's (or organization's) contributions
amount to a nontrivial percentage of global emissions; {\em everyone} will
need to contribute, to achieve the global 40\% reduction needed by 2030. To
get others on board, we must set an example. Indeed, it is especially
important for scientists---who, as a community, have been instrumental in
raising the alarm about climate change---to not appear to be exempting
themselves from cutting back.  But they must do so in a smart way, or else
risk disproportionately harming the progress that science can bring.

\SECTION{Putting a Price on Emissions}

Fortunately, there is a well-established mechanism for comparing apples and
oranges---indeed, for comparing almost anything!  Economists call it ``the
magic of markets.''  By mapping choices and preferences in disparate domains
onto a common scale (i.e., money), economic markets function as
massively distributed computations that optimize costs and benefits across
vastly different activities.\bcp{Give a simple, concrete example.}  For this
reason, many policy experts advocate addressing climate change through some
form of carbon pricing, which places a monetary cost on greenhouse gas
emissions, sending a clear market signal that changes aggregate
behavior.\bcp{See
  https://kleinmanenergy.upenn.edu/policy-digests/why-carbon-pricing-falls-short
  for lots of good phrasing and citations.}  Exactly how the price of carbon
should be set, and what should be done with the raised funds, is a
challenging policy question, but it has a clear goal: It should be chosen so
that the world arrives at zero emissions within the timeframe that climate
scientists tell us is safe.

While a global \COtwoE{} market is the ultimate goal, we do not need
to get there in one fell swoop.
Some countries\bcp{e.g.?}, US states (e.g., California), universities
(e.g., UMD and UCLA), and companies (\mwh{Jane Street?}) have already
begun to adopt internal carbon pricing policies and markets, to push
their own behavior closer to a sustainable equilibrium.  We argue that ACM
should do the same thing. Here's how it might start.

\SECTION{A Carbon Market within ACM}

The main idea is to for ACM to enforce a carbon tax on conference
budgets. At present, ACM charges an ``allocation'' percentage of
12\% on total revenue. It can also impose a tax on
the \COtwoE{} emissions induced by the conference. This tax has two
benefits. First, it creates an incentive for SIGs and conference organizers to
lower total costs by reducing emissions, so they can pay less
tax. They could do this in many ways, such as the ones outlined in our 
report. Second, ACM can use the extra revenue for ``good works.'' Such
good works could directly offset the carbon that paid for them,
or they could provide some other benefit, such as defraying open access
publishing costs.  \bcp{I foresee a tricky legal question of whether it's OK for
  ACM to charge such a tax simply for ``good works.''}

Getting this idea rolling requires three things: (1) processes to
measure \COtwoE{} used by a conference, for assessing the tax; (2) setting
a tax to the conference budget for the \COtwoE{} it consumed; and (3)
allocating and disbursing collected funds to good works. 

For the first, we can start by estimating \COtwoE{} emissions due to
conference travel by analyzing the conference's registration data. We
worked with the ACM SGB to develop a \COtwoE{} calculator that takes such
registration data and estimates the \COtwoE{} cost borne by each
individual's travel to the conference. It could become part of
standard practice to run this calculator when closing out the
conference, to assess the tax. \mwh{other \COtwoE{} costs?}  \bcp{But the
  conference planners would need at least an estimate of what the tax was
  going to amount to, to avoid a budget shortfall.}
%
Moreover, intrepid SIG leaders and conference organizers can analyze
the gathered data to look for opportunities for reduction. For
example, we used the calculator to compute the per-capita and total
\COtwoE{} costs of several SIGPLAN conferences over the last few years.
\mwh{Discuss.}
\bcp{Put the discussion of Figure 1 (and an accompanying figure
showing distance traveled to several instances of the same conference) here;
this illustrates the difficulty and the need for careful data gathering and
analysis.}

\begin{figure}[t]
\centering
\includegraphics[width=6in]{SIGPLAN-confs.png}
\includegraphics[width=6in]{ParticipantsOriginAll.png}
\caption{Top: carbon footprint per participant for travel to recent SIGPLAN
  conferences\iflater\bcp{Get better file from the other paper, or get
    better version of this one from Crista.}\fi Bottom: origin of participants for each conference.} 
\label{fig:confs}
\end{figure}

How should we set the price of a conference's \COtwoE? There are several
possibilities. One idea is simply to adopt the price from an existing
market, such as California's; between 2012 and 2018, the price of a
ton of \COtwoE{} fluctuated from about \$12 to \$23~\cite{CACO2}. Another idea is to
match the price to the equivalent in \emph{carbon
  offsets}~\cite{CarbonOFfsetReport}. Offset providers use raised funds
to support an activity (e.g., installing methane capture devices on
landfill sites or buying fuel-efficient stoves to replace open-fire
cooking in poor communities) that is {\em certified} (by yet another
organization) to permanently avoid emitting an amount of greenhouse
gases specified. Current prices vary widely around an average of around
\$15 per ton, depending on the vendor, the nature of the offset
project, the region of the world where it will be implemented, etc.
On the other hand, the true
``social cost'' of a ton of carbon is variously estimated to be much
higher than the above---anywhere from \$40 to \$400 per
ton~\bcp{citation}---so current pricing may already be on the low
side.

% Point from Yannick email: CO2 price could be cost to put CO2 out the air, or the societal cost:
% One can seek for the market to converge toward the price that would allow for funding the necessary infrastructure to be put in place in order to suck out of the atmosphere 1 ton of co2-e, say through tree planting, carbon-capturing agriculture or what not.
% One can seek for the market to evaluate the societal cost of the emission, evaluating health-related impacts, floods and what not, and putting a price on this damage.

However we estimate the price, we could imagine setting it a little low at the
outset, perhaps based on analysis of existing conference \COtwoE{} usage, and then raising it
over time to give organizers time to adapt.  

Lastly, having collected the carbon tax from the conference budget,
what should ACM do with it? One obvious thing to do is to buy \COtwoE{}
offsets.\footnote{Some ACM conferences have already started to
  experiment with voluntary carbon offset fees at registration time,
  but with little luck.}
%
Carbon offsets are not a long-term solution to global warming, and the
details matter, but at this point we believe there are options that do
far more good than harm.
%
Many organizations (including companies such as Google, Dell, Microsoft,
General Motors, Delta Airlines, Lyft, and Expedia, numerous universities,
and a number of academic societies) are already buying carbon offsets.
%
One superficial impediment is that most organizations do not currently have
accounting procedures in place to deal with them, and the piecemeal
political and bureaucratic work of designing and implementing a separate
policy for each organization can consume significant energy. \mwh{Not
  sure the previous is the right thing to say? Basically: Need to work
  out the legality. Cite offset report?}

Another thing is to do \mwh{finish off. Here, talk about other good
  works, like open access fees, developing technology for CO2e
  reduction of hosting conferences, perhaps pay for remote presence,
  ...} 

\SECTION{Conclusion}

The climate crisis is too large and too urgent to leave to world leaders to
address at their own pace: organizations at every scale must confront
their own contributions, raise awareness among their members, and begin
establishing new ways of doing business in the lower-carbon future that is
coming.
\bcp{A sentence restating our proposal.}

% For ACM, this especially means confronting the carbon-intensive
% nature of our conferences.
% %
% You can help accelerate this process in a number of ways: 
% \begin{itemize}
% \item {\bf Get involved} in raising awareness about this issue in the
% organizations you are a part of.  \bcp{Does the rest of this smack of
%   incrementalism?  Maybe delete it.  ``For example, ask the organizers of
%   the conferences you attend what they are doing about reducing or
%   offsetting emissions from participants' travel.''}
% \item {\bf Connect} with others that are engaging with these issues---in
% your workplace, in your other communities, and within ACM (e.g., on the
% {\tt acm-climate} mailing list~\cite{acm-climate}).
% \item {\bf Analyze} your organization's carbon-emitting activities to
% understand how to reduce its footprint now and prepare for the coming era of
% expensive carbon.
% \end{itemize}

\bcp{Things still missing:
  \begin{itemize}
  \item citations for the claims we make (and the carbon-offsets-are-bad paper
  mentioned by the reviewer)
  \item Go through all the comments and ideas below (including the CACM
  reviews) one last time
  \item Find a place for this: {\tt acm-climate} mailing list~\cite{acm-climate}
  \item Think about adding headings, per Bishop's recommendation

  \item Write a cover letter
  \begin{itemize}
  \item We tried hard to remove all incrementalism and make it much more
  hard hitting
  \item However, we did keep the suggestion of carbon offsets, hopefully
  in a more convincing / palatable / better argued form
  \item what else?
  \end{itemize}
  \end{itemize}
}

\bibliographystyle{unsrt}
\bibliography{local}

\iflater
\newpage
\section*{Permissions for figures}

[1] Creative Commons license

[2] ``Reproduction of a limited number of figures or short excerpts of IPCC material is authorized free of charge and without formal written permission provided that the original source is properly acknowledged, with mention of the name of the report, the publisher and the numbering of the page(s) or the figure(s).''

[3] Authors

\section*{Old Text}

We are the members of an {\em ad hoc} Committee on Climate Change within
ACM's Special Interest Group on Programming Languages (SIGPLAN) that was
formed in 2016 to consider possible strategies for mitigating the emissions
of our conferences \cite{ClimateCommitteReport}. Among the ones we find 
attractive are making it easier to use carbon offsets to mitigate aviation
emissions for conferences, choosing conference locations with an eye to
minimizing emissions, co-locating or merging conferences, virtualizing
program committees, and supporting remote participation.

Buying carbon offsets to reduce the impact of air travel
\cite{CarbonOFfsetReport} is an ``easy win'' that conferences should
implement now.  The idea of carbon offsets is to pay somebody to plant
trees, capture methane (from landfills or cattle herds, for example), buy
cookstoves to replace inefficient open-fire cooking in poor communities, or
carry out other actions that remove greenhouse gases from the air or prevent
their emission. At the moment, the average air travel to an international
conference can be offset by paying around \$30 \cite{CarbonOFfsetReport}.
ACM could streamline the purchase of such offsets and make offsetting the
default by folding them into its conference registration fees.

However, carbon offsetting is only a short-term mitigation, since the
potential supply of offsets is far less than the potential
demand~\cite{SEI-Report}. We must also begin significantly reducing
emissions, which means reducing the distance traveled by the sum of all
conference participants.

One way to do this is to choose conference locations that are close to
the center of mass of their likely participants.  For example, the
SIGGRAPH conference on computer graphics
locates in Los Angeles every other
year---an easy call, as this is near both an international airport and a major hub
of the worldwide entertainment industry. For other research areas, the
effects of location choice are less obvious. For example,
Figure~\ref{fig:confs} plots the 
locations of the four major SIGPLAN conferences over the past five years,
colored according to the estimated average carbon footprint per
participant. The figure shows significant differences across
conferences and across locations---indeed, a factor of two difference
between the 
lowest and highest. Why is this? One reason might be that the locus of a
conference's population is closer to one location than another, and changing
locations reduces the total cost. Another might be the presence of a local
population that will only attend if the conference is very close to
them. Our preliminary analysis suggests it may be both (and that deeper
analysis is needed to understand the situation better). The larger question
is how to use this kind of information to choose the best locations, over
time, for a diffuse population.

Another natural idea is to co-locate or consolidate conferences.  This is
already standard practice in some communities---e.g., SIGGRAPH, SC, and CVPR
are all ``the'' conference for their areas.  More communities---for example,
the programming languages and security communities, each with four
main conferences every year---could consider following suit.  A different
twist on this idea would be to hold a single conference in two locations
simultaneously (e.g., Boston and Paris), with a video link between the
sites. Switching to virtual program committee meetings for conferences is
another obvious step.\iflater \mwh{data about surprising connection to conference's
  overall footprint?}\bcp{Or leave it for next round of revisions when we
  may have done more/better analysis?}\fi{}  Indeed, this switch is already
happening: we found in a recent poll of ACM SIG chairs that, across ACM, a
majority of the SIGs' main conferences now use online PC meetings rather
than in-person ones.

Another approach is allowing some conference attendees to participate
remotely, saving both emissions and travel time.  Indeed, SIGPLAN already
livestreams all its four main conferences, while SIGACCESS and SIGCHI have
experimented with telepresence robots at their main
conferences~\cite{CHI-remote,Neustaedter}. \iflater\bcp{Next sentence breaks
the flow a bit.  Cut it?}\fi{}Journal-first initiatives, which
make conference attendance optional, can also help.  Looking further
ahead, some communities are already experimenting with fully virtual
% \mwh{reality?}
% BCP: Seems redundant with "fully"
conferences~\cite{OS}.  ACM members can contribute to the
development of technologies for connecting people effectively at a distance.

Each of these strategies will have some beneficial effect on emissions.  
Each may also have downsides. For example, locating conferences
near their existing centers of mass may discourage holding them in
countries like India and China where the community is still developing,
while virtual PC meetings and remote conference attendance may reduce
opportunities for informal connections and mentoring.  Getting beyond easy,
short-term solutions and fat-trimming exercises requires balancing competing
concerns; this will demand significant  
discussion, compromise, careful analysis of data, and creative
experimentation.  
%
\iflater\bcp{Added remainder of paragraph.  I'm not sure the two sentences work
  together very well, but I can't decide which to keep!  (Mike likes them
  both.)}\fi%
It will also involve questioning some fundamental assumptions about how
research gets done---e.g., that more and bigger conferences are better, and
that personal success is (positively) correlated with time spent in
airports.
%
To thrive over the long haul, we need to begin imagining what a zero-carbon
professional society will look like.


\section*{More Discussions:}
\begin{itemize}
\item Our notes:

Remove the IPCC image and mark the other “uninformative” image as up to the design department (i.e., we’re happy to keep it or drop it or change it to something else)

The map of registration data should be improved at least by reformatting and possibly by better data analysis, if we can get any of that done in time.  I need to schedule a meeting with Aalok and Yannick about that.

Some of the more “incremental” ideas can be removed — we included some that we ourselves were not very excited about.

We can make clearer that carbon offsets are controversial, and why

We can use the space gained by some of the above to add some discussion at the end about:

Why it’s important that ACM be serious about making serious cuts (i.e., because it gives us moral authority and sets a good example for other parts of society), even though in some sense anything that we (or any individual or “small” group does is a drop in the ocean).

The fact that ACM should also be thinking further ahead, which may mean fundamentally questioning its entire business model, along with a call to start this conversation.

\item New comments from Judith Bishop:

1. Figure 3 is unclear. I did not understand it. Maybe it is missing a key, or the circles are too small. A rework would be very beneficial.

2. All articles benefit from headings. It reveals the structure of the paper at a glance and assists in avoiding repetition. Think about some headings. Have a look at some recent Viewpoints papers.

3. The title is rather opaque. The article is mostly about conferences, so maybe that should be in the title. Viewpoints likes catchy titles, so something more personal such as "Your conference trip and climate change" might get more readers.

4. The article does not mention the journal publication alternative to conferences. We all know that computer scientists fight fiercely for the status of their conference papers on their CVs. Maybe climate change will be the tipping point that will cause them to become more like their colleagues in other disciplines and publish in journals, with just one conference a year, to meet colleagues and catch up. From my side, I do think this aspect should be mentioned.

\item Old comments from Andrew Chien:
\begin{quote}
So, should we really be so conference intensive? (we seem to be far more so than other disciplines)

If we are going to have SIG conferences, as a staple of CS, do they actually need to be face to face?  (could
make the f2f ones less frequent)  What would the scientific impact be?

We seem to have wired into the economics of our conferences that "bigger attendance is better".  this supports
the fixed costs, makes the hotels happy, etc.  Can we make conferences where smaller in-person attendance, but
larger total engagement (whatever that means... perhaps virtual or after the fact interaction) is maximized?

What does a zero-carbon professional society look like?  (if you peek at my zccloud.cs.uchicago.edu page, you'll
see that I don't mean offset, but rather actual zero-incremental carbon emissions)

If you've already thought about all of these, then that's great.  If not, please do take them as a set of potential
challenges to consider (all voluntary... as I intend them only as suggestions).
\end{quote}
\item 
About Crista's figure:
BCP: Would be great to supplement with a more refined analysis of registrations for each of these conferences; then we could add a "center of mass" dot for each of the conferences to the map. This would still not tell the whole story, since it's possible (likely) that the registration totals would have been different if the conferences had been located differently. But it would help tell the story.
\item 
Also about Crista's figure:
\begin{itemize}
\item 
BCP: We should quantify what the red and blue dots mean -- i.e., give readers an idea what overall difference a change of location would make to carbon footprint.
\item Mike: The dot colors mean per-capita footprint. Could we make their size represent the total footprint? Far-flung places might be high per-capita, but lower overall.
\item Crista: The order by total footprint is very similar to the normalized one. Here are the 5 top offenders by that criteria: ICFP-Japan, POPL-LA, POPL-SD, SPLASH-Vancouver, PLDI-Barcelona. There are some changes in the order of the conferences in the middle. I hesitated which plot to draw (each has positives and negatives) but decided this one would be less problematic. I don't think we have space for two, and overlaying the 2 data series will be confusing...
\item BTW, there's a LOT more that can be said about this data... But I think that will distract from the main message of this article. We need to chose whether this 2-page article is about the data analysis of our conferences or about a call for action. Maybe we can write a data analysis paper, upload it on arXiv, and link to it from this article?
\item
For example: POPL-Paris was the absolute winner by all criteria: it had the highest number of participants, the second-lowest normalized carbon footprint, and was mid-point in total carbon footprint, even with all those participants.
This is just one tidbit; there are many more interesting facts in this data.
\end{itemize}
\item 
From Michael Coblenz:
What about research on energy-efficient computing? The impact of this research might be far more significant than that of mitigating our direct carbon emissions, and this is more in line with the core skills of the community.
Also, what about research on telepresence and virtual meetings? There's obviously a lot of work in this area, but we may need more in order to (a) let us effectively trade off options, e.g. virtual PC meeting vs in-person; (b) improve the state of the art to make remote meetings more effective.
Consider also discussing the diversity implications on travel. Reducing travel expectations might make the field more attractive to women, who may face higher social expectations to stay home (and are more likely to be single parents). Surely there is a good article to cite here. I found https://journals.sagepub.com/doi/abs/10.1177/0950017006066999 but maybe there is something better. I'm looking…
\item 
Consider citing the new IEA report as another piece of evidence that action is needed
\item 
From Sophia: Does the article need an abstract to "set the reader's expectations"? I expected that there would be a list of actions in the end, or recommended resolutions. Instead, the article is more about what some communities do, and where individuals may be involved.  Perhaps say: "In this article we give a short overview of how different SCM communities tackle this problem, what thought are relevant to it, and conclude with suggestions on how to get involved". It breaks the flow, but it clarifies what the article is about.
\item 
Is the citation for emissions by region CO2 or CO2e?
\item 
Is there a ``more authoritative'' source for price of carbon offsets than
our carbon report?
\end{itemize}
\item 
Check where we've used ``carbon'' (or CO2) and where ``greenhouse gases''.  Gloss CO2 or CO2e as appropriate.
\fi

% \end{multicols}
\end{document}
